본 대규모 데이터분석 특강 프로젝트에서 우리는 슈퍼 마리오 브라더스 게임을 강화학습을 통해 컴퓨터가 학습하고 최종적으로 높은 점수로 스테이지를 클리어 하는 것을 목표로 한다.

슈퍼 마리오 브라더스에서 사용 가능한 동작은 총 6 가지로 구성되어 있으나 서로 다른 키를 조합하여 입력하는 것이 가능하므로 실제로 사용 가능한 동작은 총 16개로 구성되어 있다. 해당 게임은 agent인 마리오가 목적지에 도달하거나, 적에게 부딪히거나, 제한 시간 내에 목적지에 도달하지 못하면 stage가 끝남으로써 episodic한 요소를 가지고 있다. 

따라서 해당 게임을 강화학습을 통해 클리어하기 위해 실제 인간에게 주어지는 것처럼 현재 화면을 입력 값으로 받고 이를 통해 최적의 동작(action)을 찾아내는 모델인 convolutional neural network (CNN)을 구축하는 것을 기반으로 다양한 시도를 할 것이다.

프로젝트 기간내에 효율적으로 시간을 분배해 최적의 결과를 도출하기 위해 각 구성원의 역할을 개발환경 세팅, 기존 방법 학습, 새로운 방법 연구, 제안 방법 개발과 같은 방식으로 세분화한다.

해당 프로젝트 제안서의 구성은 다음과 같다. 
2 장에서는 본 프로젝트에서 풀려고 하는 슈퍼 마리오 브라더스 게임을 학습하고 목표를 달성하기 위해 도움이 될만한 기존 논문의 메인 아이디어를 간략히 설명하고 어떻게 프로젝트 구현에 사용될 수 있을지에 대해 설명한다. 
또한 논문의 메소드를 본 프로젝트에서 사용함에 있어 제약사항 혹은 문제가 될 여지가 있다면 이러한 부분에 대해 서술한다. 
3 장에서는 슈퍼 마리오 브라더스 게임에 대해 간략히 설명하고 강화학습을 진행하는데 필요한 state, action, 그리고 reward이 게임에서 어떻게 구성되어 있는지에 대해 설명한다. 
4 장에서는 본 프로젝트에서 사용할 라이브러리에 대한 설명과 각 구성원의 개발 일정에 따른 역할에 대해 설명한다. 
마지막으로 5 장에서는 해당 제안서에서 설명한 부분들에 대해 정리한 뒤 본 프로젝트 제안서를 마무리하는 것으로 구성되어 있다.