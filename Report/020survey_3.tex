\subsection{Deep Deterministic Policy Gradient}
Timothy P, et. al. 이 제안한 Deep Deterministic Policy Gradient (DDPG)~\cite{DDPG}는 state와 action을 입력으로 받아 각 action에 대한 value값을 산출해내는 Q-function을 학습시키는 방법이다.
DDPG의 핵심 동작은 아래와 같은 식으로 나타낼 수 있다.
\begin{align*}
	a^*(s) &= arg \underset{a}{max} Q^*(s,a) \\
\end{align*}
%\begin{align*}
%	h_t &= f(h_{t-1},x_t;W,b) \\
%		&= \tanh{(W_{hh}h_{t-1}+W_{hx}x_t+b_h)} \\
%		&= \tanh{(W_{h}[h_{t-1},x_t]+b_h)}
%\end{align*}
여기서 Q-function Q 는 action a와 state s를 입력으로 받아 value를 판별해주므로 최대 value를 가지는 action을 선택하면 최적화된 policy를 구할 수 있다.

DDPG는 다음과 같은 특징을 가지고 있다.
\begin{itemize}
	\item off-policy 방법
	\item continuous action space에서만 동작하는 방법
	\item 병렬화된 학습 방법을 제공하지 않음
\end{itemize}
본 프로젝트에서 학습하기로 한 게임인 슈퍼 마리오 브라더스는 discrete한 action space를 가지므로 DDPG 알고리즘을 바로 적용할 수는 없다.
하지만 DDPG에서 학습을 돕기 위한 주요 아이디어를 얻어와 우리 모델을 설계하는 데 활용할 수 있다.

\paragraph{\textbf{Replay Buffer:}}
DQN에서 소개된 바 있는 Replay Buffer를 DDPG 방법에서도 사용한다.
Replay Buffer의 핵심 아이디어는 학습에 사용되는 샘플들의 상관관계를 줄여서 overfitting을 막는 것이다.
한 에피소드의 인접한 타임라인에서 채취된 샘플들은 밀접한 상관관계를 가진다.
Replay Buffer로 과거 경혐을 저장하고 랜덤으로 샘플을 추출하여 학습에 활용하면 이러한 상관관계를 줄이는 효과가 있다.
DDPG 는 off-policy 를 위한 방법이므로 Replay Buffer의 크기가 DQN에 비해 커질 수 있다.
GPU를 통한 병렬처리를 통해 하드웨어의 성능을 최대로 활용하기 위해서 DDPG는 Replay Buffer에서 채취한 샘플들을 minibatch 형태로 학습한다.
\paragraph{\textbf{Soft Updating Target Network:}}
Target Network를 이용한 weight 업데이트 방식은 DQN에서 소개한 아이디어와 같다.
하지만 Actor-Critic에 맞추어 weight를 바로 업데이트하지 않고 일정 비율로 업데이트될 값을 아래와 같이 절감한다.
\begin{align*}
	\theta^{'} \leftarrow \tau\theta + (1 - \tau)\theta^{'}
\end{align*}
여기서 $\tau << 1$ 이다.
이는 Target value의 값이 천천히 변화하도록 제한하는 것으로 학습이 안정되는 것을 도와준다.
하지만 전반적인 학습 속도를 늦추는 경향이 있다.

