%Put related works here.
We review previous researches of the news recommendation system in content provider companies.

Okura et al.~\cite{Yahoo2017} introduced their research about news recommendation system serviced by Yahoo Japan.
They prepare triple of articles (an article, an article in the same category with first, and in the other category with first) to train the VAE-based (variational autoencoder) model to make article representation closer when articles are in the same category.
This method increases understaing of fresh articles with category meta-information to relieve the cold-start problem.
And they use GRU-based model to provide a personalized recommendation by capturing latent features for the user in the session.
But for every 2 weeks, they have to retrain the whole model from the beginning to preserve the quality of recommendation as fresh articles are stacked in data.
%We introduce previous researches for the news recommendation system with the session-based data.

%Okura et al.~\cite{Yahoo2017} introduce their works to improve the news recommendation system serviced by Yahoo Japan.
%The main idea of their method (\compYahooName) is generating the article representation using the category information to resolve the Cold-Start problem.
%They prepare the triple of the article, the article with the same category, and with the other category.
%

%Okura et al.~\cite{Yahoo2017} introduced the news recommendation method for the front page in the yahoo japan.
%Their method considers the representation of the article and user separately, they let these representation vectors have the same dimension so they do the simple inner product between the article and user vectors to calculate the relevance of them.
%To resolve the cold-start problem, they do the additional vector embedding for articles to let vectors have category related feature.
%Their idea comes from the variational autoencoder (VAE)~\cite{VAE} which has the internal hidden state the dimension of which is lower than original article's.
%They add the loss function to reduce the similarity of hidden state between articles in the same category and to increase in the other categories.
%They generate user representation from the browsing history of each user to increase the relevance factor with clicked articles and to reduce with not clicked.
%The main weakness of their method is that the entire model needs to re-calculate all of the representations to accept the new distribution of data as time goes on.

NAVER is a popular web portal service in South Korea which provides hundreds of news articles every day.
Park et al.~\cite{Naver2017} published the paper about their improvements of news recommendation in NAVER.
They use category information to relieve the cold-start problem but only about 40\% of articles have category data in the meta-information field.
They propose CNN-based (Convolutional Neural Network)~\cite{CNNHinton} model to predict the category of an article from its sentences.
They accumulate changes of category preference in the session with a decaying method and use it when scoring candidate articles by categories of candidates.
But this method does not capture long-term preferences well from previous sessions.
In contrast to their result, according to the research by Setty et al.~\cite{SettyHistory}, history-based information from previous sessions improves the accuracy of the recommendation system.

%NAVER is the most popular web portal service in South Korea which provides hundreds of news articles every day.
%Park et al.~\cite{Naver2017} introduced the news recommendation system developed in the NAVER.
%In the real world dataset, only about 40\% of articles are classified as a category information which should be inputted by the reporter.
%They proposed the convolutional neural network based category classifier to predict unknown category information.
%And they proposed the history-based recurrent model to capture the long-term preferences of users, but they just use inputs from the previous session to put them to RNN model.
%So their accuracy is even worse than baseline methods.
%In contrast to their result, according to the research by Setty et al.~\cite{SettyHistory}, the history-based information improves the accuracy of the sequence-aware recommendation system.

%Chung et al.~\cite{GRU} proposed the Gated Recurrent Unit (GRU) as the abbreviated structure of the LSTM cell.
%Hidasi et al.~\cite{GRU4Rec} published their research about using the multi-layered GRU (\compGruRecName) for the recommendation system with the session-based input.
%They show that the multi-layered GRU outperformed the other competitors which are used for the recommendation task.
%
%The Naver is the web portal service in South Korea.
%Park et al.~\cite{Naver2017} published their research about the news recommendation system tested with the news content service on the front page of the Naver.
%In their method (\compNaverName), they aggregate the category information of news article to the model when training.
%Because of the sparsity of the meta-information, they introduced the Convolutional Neural Network (CNN)~\cite{CNNHinton} based method to predict the empty entry for the category in the data.
%They utilize the category information when ranking the candidate articles that they keep the preference about the category in the session with the exponential decay method.
%They additionally score the candidate articles using the category of the candidate articles and the preference of category in the session.
%
